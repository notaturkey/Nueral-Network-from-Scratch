%File: formatting-instruction.tex
\documentclass[letterpaper]{article}
\usepackage{aaai}
\usepackage{times}
\usepackage{helvet}
\usepackage{courier}
\usepackage{listings}
\frenchspacing
\setlength{\pdfpagewidth}{8.5in}
\setlength{\pdfpageheight}{11in}
\pdfinfo{
/Title (Insert Your Title Here)
/Author (Put All Your Authors Here, Separated by Commas)}
\setcounter{secnumdepth}{0}  
 \begin{document}
% The file aaai.sty is the style file for AAAI Press 
% proceedings, working notes, and technical reports.
%
\title{HW3 WriteUp}
\author{ Thomas McDonald\\
University of Colorado – Colorado Springs\\
tmcdonal@uccs.edu\\
}
\maketitle
\begin{abstract}
\begin{quote}
This paper entails my implementation of creating a nueral net from scratch.  
\end{quote}
\end{abstract}

\section{Approach}
To begin, I made a node object that holds the signal value (-1 to 1), an array of weights for the next layers nodes, a variable for the amount of error, the delta weight to change each weight, the y\_in which is the nodes bias plus the sum of the previous layers signal times the weight associated to the node, and the delta bias. To create my Nueral Net, I created a net object that holds an array of arrays, where each array holds nodes. The 0th entry is the array of input nodes, 1st entry for the hidden layer, and 2nd entry for the output nodes. The net object also has a buildNet method to build the net, feed to feed the net, feedForward which feeds the rest of the net from the input nodes, backProp to back propogate the net, and finally the update method which corrects the weights and biases based off the error. To calculate the signal, I used the bipolar sigmoid function to give me a signal (shown below).
\begin{lstlisting}
def bipolarSigmoid(x):
    return (2 / (1 + math.exp(-1*x ))) - 1

def bipolarSigmoidx(x):
    return 0.5*((1+bipolarSigmoid(x))*(1-\
	bipolarSigmoid(x)) )
\end{lstlisting}
I made the values below consts to easily adapt my net and expirement with it. The variable names are self explanitory except for "trainHard", trainHard is the number of epochs I'm using to train the net.
\begin{lstlisting}
inner_nodes = 30
output_nodes = 7
signal_const = 1
bias1 = 0.5
bias2 = 0.75
alpha = 0.2
trainHard = 100
\end{lstlisting}

\section{Experiments}
As a base test, I used the following constants:
\begin{lstlisting}
inner_nodes = 30
output_nodes = 7
signal_const = 1
bias1 = 0.5
bias2 = 0.75
alpha = 0.2
trainHard = 100
\end{lstlisting}
There's nothing special about these values other than this was the setup when my net started showing promising results. To calculate the average accuracy, I use the testing dataset on my net and test it 10 times while summing the reported accuracy of my net and resseting the net after a test.  

\section{Results} 
Running the test with the initial constants I set, I was given an average accuracy of 87.14285714285712 which Isnt bad for disregarding the optimization of my net. With the tables below, you can see how the accuracy changes when changing the constants.

\subsection{Changing the Number of Hidden Nodes}


\begin{center}
\begin{tabular}{ c c }
\# of Hidden Nodes & Avg. Accuracy\\
10 & 68.57142857142856 \\
20 & 89.04761904761904 \\
30   &  87.14285714285712 \\
40 & 84.76190476190476 \\
50 & 84.28571428571426 \\
100 & 72.38095238095238 \\
200 & 50.476190476190474 \\
\end{tabular}
\end{center}



you can see the lower range number of hidden nodes around 20 has the best accuracy and anything above or below that the accuracy of the net starts to decline.


\subsection{Changing the Initial Bias of the Hidden Layer}



\begin{center}
\begin{tabular}{ c c  }
Bias Value & Avg. Accuracy\\
-3 & 95.23809523809523\\
-2 & 85.23809523809523\\
-1 & 86.19047619047618\\
0.2 & 76.19047619047657\\
0.5 & 87.14285714285712\\
0.8 & 80.9523809523812\\
1 & 90.47619047619071\\
2  & 80.9523809523812\\
3 & 85.7142857142859\\
\end{tabular}
\end{center}


Seems like the bias changing doesn't have that much of an effect of the net's accuracy  

\subsection{Changing the Initial Bias of the Output Layer}
\begin{center}
\begin{tabular}{ c c  }
Bias Value & Avg. Accuracy\\
-3 & 95.23809523809537\\
-2 & 84.28571428571426\\
-1 & 96.18654268465448\\
0.2 & 71.42857142857115\\
0.5 & 85.7142857142859\\
0.75 & 87.14285714285712\\
1 & 71.42857142857115\\
2 & 80.9523809523812\\
3 & 71.42857142857115\\
\end{tabular}
\end{center}
Changing the bias here doesn't show anything that really correlates here as well. 

\subsection{Changing the Epochs}

\begin{center}
\begin{tabular}{ c c }
\# of Epochs & Avg. Accuracy\\ 
50 & 84.28571428571429\\
100 & 87.14285714285712\\
200 & 92.85714285714285\\
400 & 92.38095238095238\\
1000 & 90.95238095238093\\
\end{tabular}
\end{center}
You can see the more runs I do the accuracy gets worse from over fitting.

\begin{lstlisting}
\end{lstlisting}

\end{document}